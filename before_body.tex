% Seleciona o idioma do documento (conforme pacotes do babel)
%\selectlanguage{english}
\selectlanguage{brazil}

% Retira espaço extra obsoleto entre as frases.
\frenchspacing 

% ----------------------------------------------------------
  % ELEMENTOS PRÉ-TEXTUAIS
% ----------------------------------------------------------
  
  %---
  %
% Se desejar escrever o artigo em duas colunas, descomente a linha abaixo
% e a linha com o texto ``FIM DE ARTIGO EM DUAS COLUNAS''.
% \twocolumn[    		% INICIO DE ARTIGO EM DUAS COLUNAS
%
%---
  % página de titulo
\maketitle

% resumo em português
% \begin{resumoumacoluna}
% Conforme a ABNT NBR 6022:2003, o resumo é elemento obrigatório, constituído de
% uma sequência de frases concisas e objetivas e não de uma simples enumeração
% de tópicos, não ultrapassando 250 palavras, seguido, logo abaixo, das palavras
% representativas do conteúdo do trabalho, isto é, palavras-chave e/ou
% descritores, conforme a NBR 6028. (\ldots) As palavras-chave devem figurar logo
% abaixo do resumo, antecedidas da expressão Palavras-chave:, separadas entre si por
% ponto e finalizadas também por ponto.
% 
% \vspace{\onelineskip}
% 
% \noindent
% \textbf{Palavras-chave}: latex. abntex. editoração de texto.
% \end{resumoumacoluna}

% ]  				% FIM DE ARTIGO EM DUAS COLUNAS
% ---

% ----------------------------------------------------------
% ELEMENTOS TEXTUAIS
% ----------------------------------------------------------
\textual
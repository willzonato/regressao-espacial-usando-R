\usepackage{lastpage}			  % Usado pela Ficha catalográfica
\usepackage{indentfirst}		% Indenta o primeiro parágrafo de cada seção.
\usepackage{color}				  % Controle das cores
\usepackage{graphicx}			  % Inclusão de gráficos
\usepackage{microtype} 			% para melhorias de justificação
\usepackage{hyperref}       % para inserção de hiperlinks
\usepackage{float}          % para ancoragem das figuras no local exato.
\usepackage{fontspec}       % para uso de fontes do MS-Windows.
\usepackage{csquotes}       % para uso de aspas duplas com a função \enquote
\usepackage{etoolbox}

\setmainfont{Times New Roman}
\fontsize{10.5}{10}\selectfont

\setlrmarginsandblock{2.5cm}{2.5cm}{*}
\setulmarginsandblock{2.5cm}{2.5cm}{*}
\checkandfixthelayout
% ---
		
% ---
% Pacotes de citações
% ---
\usepackage[brazilian,hyperpageref]{backref}	 % Paginas com as citações na bibl
\usepackage[alf]{abntex2cite}	% Citações padrão ABNT

% ---
% Informações de dados para CAPA e FOLHA DE ROSTO
% ---
 \titulo{Regressão Espacial com o uso do R}
 \autor{Gabriel Zanela\thanks{gazela@gmail.com}}
 \local{Brasil}
 \data{2017}

% ---

% ---
% Configurações de aparência do PDF final

% alterando o aspecto da cor azul
\definecolor{blue}{RGB}{41,5,195}

% informações do PDF
\makeatletter
\hypersetup{
     	%pagebackref=true,
		pdftitle={\@title}, 
		pdfauthor={\@author},
    	pdfsubject={Modelo de artigo científico com abnTeX2},
	    pdfcreator={LaTeX with abnTeX2},
		pdfkeywords={abnt}{latex}{abntex}{abntex2}{atigo científico}, 
		colorlinks=true,       		% false: boxed links; true: colored links
    	linkcolor=blue,          	% color of internal links
    	citecolor=blue,        		% color of links to bibliography
    	filecolor=magenta,      		% color of file links
		urlcolor=blue,
		bookmarksdepth=4
}
\makeatother
% --- 

% ---
% compila o indice
% ---
\makeindex
% ---

% ---
% Altera as margens padrões
% ---
\setlrmarginsandblock{3cm}{3cm}{*}
\setulmarginsandblock{3cm}{3cm}{*}
\checkandfixthelayout
% ---

% --- 
% Espaçamentos entre linhas e parágrafos 
% --- 

% O tamanho do parágrafo é dado por:
\setlength{\parindent}{1.3cm}

% Controle do espaçamento entre um parágrafo e outro:
\setlength{\parskip}{0.2cm}  % tente também \onelineskip

% Espaçamento simples
\SingleSpacing

% ----
% Início do documento
% ----
